\subsection{定义}

图由顶点和边组成,每条边的两端都必须是图的两个顶点,记号$G(V,E)$表示图$G$的顶点集为$V$、边集为$E$。

一般来说,图可分为\textbf{有向图}和\textbf{无向图},有向图的所有边都有方向;而无向图的所有边都是双向的。

顶点的\textbf{度}是指和该顶点相连的边的条数,顶点的出边条数称为该点的\textbf{出度},顶点的入边条数称为该顶点的\textbf{入度}。

顶点和边都可以有一定属性,而量化的属性称为\textbf{权值}。

\subsection{图的存储}

\subsubsection{邻接矩阵}

设图$G(V,E)$的顶点标号为$0,1,...,n-1$,那么可以令二维数组$G[n][n]$的二维分别表示图的顶点标号,若$G[i][j]=1$,则说明顶点$i$和顶点$j$之间有边;若$G[i][j]=0$,则说明顶点$i$和顶点$j$之间不存在边,这个二维数组$G$被称为\textbf{邻接矩阵}。另外,如果存在边权,则可以令$G[i][j]$存放边权,对不存在的边的边权设置为0、$-1$或是一个很大的数。

\subsubsection{邻接表}

把同一个顶点的所有出边放在一个列表中,那么$n$个顶点就有$n$个列表,没有出边对应空表,这$n$个列表被称为图$G$的\textbf{邻接表},记为$adj[n]$,其中$adj[i]$存放顶点$i$的所有出边组成的列表。

\subsection{图的遍历}

\subsubsection{采用DFS遍历}

采用DFS遍历即沿着一条路径遍历直到无法继续前进,才退回到路径上离当前顶点最近的还未访问分值顶点的岔道口。


\subsubsection{采用BFS遍历}


\subsection{最短路径}

对任意给出的图$G(V,E)$和起点$S$、终点$T$,求一条从$S$到$T$的路径,使得这条路径上经过的所有边的边权之和最小。

解决最短路径问题常用的算法有Dijkstra算法、Bellman-Ford算法、SPFA算法和Floyd算法。不同的算法应用于不同的题目。

\subsubsection{Dijkstra算法}

Dijkstra算法用于解决\textbf{单源最短路问题},即给定图$G$和起点$s$,通过算法得到$s$到其他每个顶点的最短距离。

\textbf{基本思想}是对图$G(V,E)$设置集合$S$,存放已被访问的顶点,然后每次从集合$V-S$中选择与起点$s$的最短距离最小的顶点$u$,访问并加入集合$S$。之后,令顶点$u$为中介点,优化起点$s$与所有从$u$能到达的顶点$v$之间的最短距离。这样的操作执行$n$次,直到集合$S$已包含所有顶点。

集合$S$可以用一个bool数组$vis[]$来实现,当$vis[i]=$true时表示顶点$ V_i $已被访问。令int型数组$d[]$表示起点$s$到顶点$V_i$的最短距离,初始时除了起点$s$的$d[s]=0$,其余顶点都赋值为一个很大的数($10^9$或者0x3fffffff,但不要使用0x7fffffff)来表示不可达。

\begin{lstlisting}
const int MAXV = 1000;	// 最大顶点数
const int INF = 1e9;	// 最大数
int n, G[MAXV][MAXV];	// n 为顶点数
int d[MAXV];	// 起点到达各点的最短路径长度
// pre[v] 表示从起点到 v 的最短路径上 v 的前个顶点
int pre[MAXV];
bool vis[MAXV] = {false};	// 标记数组true为已访问

void Dijkstra(int s) {	// s 为起点
	fill(d, d + MAXV, INF);
	d[s] = 0;	// 数组 d 初始化
	for (int i = 0; i < n; i++)	{	// 循环 n 次
		// u 使 d[u] 最小,MIN 存放该最小的 d[u]
		int u = -1, MIN = INF;
		for (int j = 0; j < n; j++) {
			// 找到未访问的顶点中 d[] 最小的
			if (!vis[j] && d[j] < MIN) {
				u = j;
				MIN = d[j];
			}
		}
		// 找不到小于 INF 的 d[u]
		// 说明剩下的顶点和起点 s 不连通
		if (u == -1)	return;
		vis[u] = true;	// 标记 u 为已访问
		for (int v = 0; v < n; v++) {
			// 如果 v 未访问 && u 能到达 v
			// 以 u 为中介点可以使 d[v] 更优
			if (!vis[v] && G[u][v] != INF && d[u] + G[u][v] < d[v]) {
				d[v] = d[u] + G[u][v];	// 优化 d[v]
				pre[v] = u;	// 记录 v 的前驱顶点
			}
		}
	}
}
\end{lstlisting}

Dijkstra算法只能应对所有边权都是非负数的情况,若边权出现负数,那么Dijkstra算法可能会出错,这时最好使用SPFA算法。

若题目给出的是无向边,只需把无向边当成两条指向相反的有向边即可。

\subsubsection{Bellman-Ford算法和SPFA算法}

Bellman-Ford算法可以解决单源最短路径问题,也能处理有负权边的情况。


\subsubsection{Floyd算法}

Floyd算法用于解决\textbf{全源最短路问题},即给定图$G(V,E)$,求任意两点$u$、$v$之间的最短路径长度,复杂度为$O(n^3)$,限制了顶点数在200以内,使用邻接矩阵来实现Floyd算法是非常合适方便的。

Floyd算法流程:

\begin{lstlisting}
for (k = 1; k <= n; k++) 
	for (i = 0; i < n; i++)
		for (j = 0; j < n; j++)
			if (dis[i][k] != INF && 
				dis[k][j] != INF &&
				dis[i][k] + dis[k][j] < dis[i][j])
				dis[i][j] = dis[i][k] + dis[k][j];
\end{lstlisting}

\subsection{最小生成树}

\subsubsection{prim算法}

\subsubsection{kruskal算法}



